Chapter 1 provides general knowledge about the problem, objectives and contribution of this paper. Section \ref{sec:problem} lays the foundation for the study by describing the core problem of designing an autonomous trading system capable of navigating the uncertainties and complexities of modern equity markets. Section \ref{sec:background} then provides essential background on the mechanics of stock trading, highlighting the roles of market sentiment, information asymmetry, volatility, and information overload in shaping price dynamics. Building on this financial context, section \ref{sec:objectives}  proceeds to define the research objectives and present a conceptual framework in which market data and textual sentiment signals are jointly processed by a trading agent. Finally, section \ref{sec:contribution} concludes with a concise enumeration of the main contributions of the thesis, setting the stage for the detailed literature review and technical development that follow.

\section{Problem Statement}
\label{sec:problem}
Stock trading entails the buying and selling of equity shares in publicly traded companies with the goal of generating profit (through capital gains and dividends) and managing portfolio risk. In practice, traders place orders on organized exchanges or over-the-counter markets, where supply and demand drive price formation. Ideally, the price of each share reflects all available information about the issuing company and market conditions. Classical financial theory (the Efficient Market Hypothesis) posits that new information is quickly incorporated into stock prices, implying that prices follow a random walk \cite{Malkiel2003}. In such a world, neither technical analysis (searching for patterns in past prices) nor fundamental analysis (evaluating a company's financials) can consistently earn returns above a simple buy-and-hold strategy. Nevertheless, real markets deviate from this ideal. The problem in automated trading arises because actual markets exhibit high uncertainty and complexity: prices swing unpredictably, often driven by investor sentiments, rumors, or insider information, and traders must process an enormous volume of data to make decisions \cite{Kyle1985, Malkiel2003, Kahneman2011, Ivashina2011}. This thesis addresses the challenge of building an autonomous trading system that can navigate these complexities and exploit patterns that human traders might miss.

\section{Background and Problems of Research}
\label{sec:background}
In order to understand the trading environment, we first consider the goals and mechanics of stock trading. Investors aim to buy shares at a low price and sell them at a higher price, or to earn dividends over time. Trading takes place on exchanges (such as NYSE or Nasdaq) and in various alternative venues, where buy and sell orders interact through an order-book mechanism. Market liquidity, meaning the ease with which shares can be bought or sold without large price impact, is crucial for efficient trading. Transaction costs, bid-ask spreads, and market regulations all influence trading mechanics, but a unifying principle is that stock prices are meant to reflect the value of a company as well as collective market expectations \cite{Brealey2022}.

A central challenge in trading is dealing with market sentiment and information asymmetry. Market sentiment refers to the overall mood or attitude of investors, which can cause price swings independent of fundamental values. Behavioral finance shows that investor mood can drive short-term fluctuations: when sentiment is high, traders may buy aggressively and push prices up; when sentiment sours, selling pressure can exacerbate declines \cite{Kahneman2011}. In fact, studies of investor sentiment indices find that shifts in sentiment can lead to increased volatility and even sudden price jumps in the short term \cite{Ung2024}. Information asymmetry occurs when some participants possess private or superior information while others do not. Classic market models demonstrate that informed traders will exploit this advantage to profit at the expense of uninformed traders \cite{Kyle1985}. Consequently, asymmetric information can amplify price volatility: empirical work shows that proxies for information imbalance (such as the probability of informed trading) tend to predict surges in volatility \cite{Watanabe2008}. In summary, investor psychology and uneven information distribution introduce noise and unpredictability beyond what purely rational models would predict.

Volatility and noise are fundamental features of financial markets. Volatility measures the degree of price fluctuation over time (a higher volatility implies larger swings and more risk). Equities often exhibit volatility clustering, which are high-volatility periods followed by calm, and vice versa, due to the arrival of news and trading dynamics \cite{Robert1982}. Moreover, “noise traders” (those trading on rumors, emotion, or irrelevant information) continuously move prices away from fundamental values \cite{DeLong1990}. In fact, theoretical work suggests that some noise is required for markets to function: without random trades, any trader seeking to exploit an undervalued stock would immediately reveal valuable information and cause the market to adjust, preventing profitable trades \cite{Grossman1980}. However, excess noise increases risk. Empirical research highlights that uncertainty and hidden information raise volatility. For example, heightened information asymmetry or sentiment shifts are associated with large price jumps and more erratic markets.

Another pervasive issue is information overload. Investors today have access to enormous amounts of data: traditional news outlets, regulatory filings, social media, analyst reports, and more. While more information in principle should help price discovery, human traders have limited capacity to process it all. Recent studies confirm that excessive information flow can actually hinder decision-making: when public information becomes overwhelming, traders require higher expected returns to compensate for the added uncertainty, and markets tend to exhibit lower trading volume and higher overall returns in the following months \cite{Bernales2023}. In effect, information overload elevates estimation risk and reduces decision accuracy. Thus, even if data is available, finding the truly relevant signals amidst the noise is a nontrivial task.

Given these challenges, traditional trading strategies have notable limitations. Technical analysis tries to exploit patterns in historical price charts to forecast future moves \cite{Murphy1999}. Fundamental analysis attempts to gauge a stock's intrinsic value from company financials and macroeconomic factors, under the assumption that prices will eventually reflect these fundamentals \cite{Penman2013}. A simple buy-and-hold approach relies on broad market growth over the long term, letting equity positions appreciate passively. Under the ideal of market efficiency, however, these strategies should not consistently outperform a passive portfolio \cite{Malkiel2003}. In practice, traders find that pure technical or fundamental approaches often fail to account for sudden sentiment shifts or new information (the “unknown unknowns”) \cite{Taleb2007}. For instance, technical rules may generate false signals during turbulent periods, and fundamental valuations can lag in reacting to real-time news. Moreover, empirical evidence shows that active trading often underperforms passive strategies. Barber and Odean report that individual investors who traded frequently earned significantly lower returns than they would have by simply holding their initial portfolios \cite{Barber2000}. This gap arises from transaction costs, timing mistakes, and behavioral biases. Collectively, these observations suggest that no single conventional method fully captures the complexity of modern markets.

\section{Research Objectives and Conceptual Framework}
\label{sec:objectives}
The overarching objective of this research is to develop an automated trading framework that can effectively navigate the challenges outlined above. Specifically, we aim to create a decision-making agent that learns to trade equities by combining traditional market data (prices, volumes, financial ratios, etc.) with alternative information sources that reflect market sentiment (news articles). The conceptual framework envisions a simulated trading environment where an agent observes market states and chooses their portfolio allocation to each stock to maximize long-term return. In doing so, the agent must infer relevant signals from noisy data streams and adapt to changing conditions.

This leads to two core goals: (i) to integrate textual and sentiment information into the trading model so that the agent can anticipate shifts in investor mood or news-driven volatility, and (ii) to employ a \gls{RL} algorithm that enables the agent to discover and exploit profitable patterns without relying on fixed rules \cite{Nevmyvaka2006}. The conceptual architecture is thus a sequential decision process: at each time step, the agent receives numerical market indicators and sentiment metrics (derived from text), processes them to form an internal state representation, and then takes an action. The market's response feeds back as the next state, along with a reward signal related to profit and risk. Over many simulated trading periods, the agent iteratively improves its policy to better capture how different signals (including investor sentiment and news) should influence trading decisions. This framework acknowledges the problems identified earlier (volatility, noise, overload) by explicitly modeling the uncertainty in market feedback and by providing a mechanism to learn which aspects of the information are most predictive.

\section{Contribution}
\label{sec:contribution}
This study makes several key contributions to the field of automated stock trading.

\begin{itemize}
  \item Developed a processing pipeline that gathers financial news articles and transforms them into structured sentiment scores linked to specific stock tickers and dates using a \gls{LLM} model.
  \item Incorporated these sentiment signals into a deep reinforcement learning framework by embedding them within the reward function of the environment.
  \item Performed extensive backtesting of the sentiment-enhanced RL agent on historical market and news data, benchmarking its performance against standard strategies such as \gls{BaH}, \gls{UP}, \gls{CORN}, \gls{ANTICOR} strategy.
\end{itemize}

\section{Organization of Thesis}
The remaining of this thesis is organized as followed.

Chapter \ref{chap:literature} delves into the existing body of knowledge relevant to this study. It will survey prior research in areas such as \gls{RL} applied to financial trading, the use of \gls{LLM}s for sentiment analysis in finance, and existing hybrid approaches. This chapter will also cover essential theoretical knowledges to establish the context and identifying the research gap this thesis aims to address.

Chapter \ref{chap:method} provides a detailed account of the research design and implementation. This includes a description of the overall \gls{LLM}-augmented reinforcement learning framework, the data acquisition process for financial market data and news articles, the sentiment extraction pipeline using \gls{LLM}s, the feature engineering techniques employed to create a comprehensive state representation for the agent, and the specifics of the reinforcement learning agent's training environment and algorithm.

Chapter \ref{chap:result} presents and analyzes the findings from the empirical evaluation of the proposed trading agent. This chapter will detail the experimental setup, including the baseline strategies used for comparison, the evaluation metrics, and a thorough discussion of the performance of the \gls{LLM}-augmented agent across the defined test period.

Chapter \ref{chap:conclusion} summarizes the entire research effort. It will reiterate the main contributions of the thesis, discuss the implications of the findings, acknowledge any limitations of the study, and propose potential avenues for future work to further advance the field.

In summary, this chapter has established the motivation for integrating \gls{LLM}–derived sentiment with reinforcement learning to improve automated stock trading. By examining the fundamental goals and challenges of trading, the chapter has framed the need for a hybrid decision‐making framework. The research objectives and conceptual design outlined here provide a roadmap for developing and evaluating an LLM‐augmented trading agent, and the stated contributions underscore the novel integration of financial theory, natural language processing, and sequential decision‐making techniques that this thesis will advance.
