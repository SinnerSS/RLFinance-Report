\section{Evaluation Parameters}
To rigorously evaluate the performance of the LLM-Augmented Reinforcement Learning (RL) trading agent, a comprehensive set of financial metrics will be employed. These metrics are chosen to provide a multifaceted view of the agent's capabilities, assessing not only its profitability but also its efficiency in managing risk and the characteristics of its drawdowns. The performance of the proposed agent will be benchmarked against several baseline strategies mentioned in Chapter 1, including Buy and Hold (BaH), Universal Portfolios (UP), Correlation-driven Nonparametric (CORN), and Anti-correlation driven Nonparametric (ANTICOR) strategies. Crucially, to isolate and quantify the specific contribution of the LLM-derived sentiment signals, the LLM-augmented RL agent's performance will also be compared against a vanilla RL agent trained solely on price and technical data, without access to sentiment input. This comparative analysis will help in determining the value added by the sentiment integration.

\subsection{Cumulative Portfolio Value}
The Cumulative Portfolio Value (CPV) represents the total market worth of the investment portfolio at the end of the evaluation period. It serves as a fundamental indicator of overall wealth generation and provides a direct measure of the strategy's ability to grow capital over time. Starting with an initial portfolio value \(P_0\), the CPV at the end of the period \(T\), denoted as \(P_T\), reflects the absolute growth achieved by the trading strategy.

\subsection{Annualized Return}
The Annualized Return is the geometric average rate of return per year over a specified time period. It provides a standardized measure of investment performance, allowing for comparison across strategies with different investment horizons. If \(P_0\) is the initial portfolio value, \(P_T\) is the final portfolio value, and \(N\) is the number of years in the evaluation period, the Annualized Return (AR) is calculated as:
\[ AR = \left( \frac{P_T}{P_0} \right)^{\frac{1}{N}} - 1 \]
This metric helps in understanding the compound annual growth rate (CAGR) of the investment.

\subsection{Sharpe Ratio}
The Sharpe Ratio is a widely used measure for calculating risk-adjusted return. It indicates the average return earned in excess of the risk-free rate per unit of total volatility (measured by the standard deviation of the portfolio's returns). A higher Sharpe Ratio suggests a better return for the amount of risk taken. The formula is:
\[ \text{Sharpe Ratio} = \frac{R_p - R_f}{\sigma_p} \]
where \(R_p\) is the average return of the portfolio, \(R_f\) is the risk-free rate of return, and \(\sigma_p\) is the standard deviation of the portfolio's excess returns.

\subsection{Sortino Ratio}
The Sortino Ratio is a variation of the Sharpe Ratio that differentiates harmful volatility from total overall volatility by using the asset's standard deviation of negative portfolio returns—downside deviation—instead of the total standard deviation of portfolio returns. It measures the excess return over a target return (typically the risk-free rate) per unit of downside risk. It is particularly useful for evaluating investments with non-symmetrical return distributions. The formula is:
\[ \text{Sortino Ratio} = \frac{R_p - R_f}{\sigma_d} \]
where \(R_p\) is the average return of the portfolio, \(R_f\) is the risk-free rate (or target return), and \(\sigma_d\) is the standard deviation of negative asset returns (downside deviation).

\subsection{Maximum Drawdown (MDD)}
Maximum Drawdown is the largest percentage decline in portfolio value from a peak to a subsequent trough during a specific period. It is a key indicator of downside risk, representing the worst-case loss an investor might have experienced had they invested at a peak and sold at a trough. If \(P(t)\) is the portfolio value at time \(t\), and the peak value before the largest drop is \(P_{peak}\) and the lowest trough value during that drop is \(P_{trough}\), then MDD is calculated as:
\[ MDD = \frac{P_{trough} - P_{peak}}{P_{peak}} \]
MDD is typically expressed as a negative percentage.

\section{Simulation Method}

