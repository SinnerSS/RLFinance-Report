\documentclass[a4paper,13pt,3p,twoside]{report}
\usepackage{scrextend}
\changefontsizes{13pt}
\usepackage[utf8]{vietnam}
\usepackage[top=2cm, bottom=2cm, left=3.5cm, right=2.5cm]{geometry}
\usepackage{xurl}
\usepackage{appendix}
\usepackage{babel}
\usepackage{xcolor}
\usepackage{outlines}
\usepackage{graphicx} % Cho phép chèn hỉnh ảnh
\usepackage{fancybox} % Tạo khung box
\usepackage{indentfirst} % Thụt đầu dòng ở dòng đầu tiên trong đoạn
\usepackage{amsthm} % Cho phép thêm các môi trường định nghĩa
\usepackage{latexsym} % Các kí hiệu toán học
\usepackage{amsmath} % Hỗ trợ một số biểu thức toán học
\usepackage{amssymb} % Bổ sung thêm kí hiệu về toán học
\usepackage{amsbsy} % Hỗ trợ các kí hiệu in đậm
\usepackage{times} % Chọn font Time New Romans
\usepackage{array} % Tạo bảng array
\usepackage{enumitem} % Cho phép thay đổi kí hiệu của list
\usepackage{subfiles} % Chèn các file nhỏ, giúp chia các chapter ra nhiều file hơn
\usepackage{titlesec} % Giúp chỉnh sửa các tiêu đề, đề mục như chương, phần,..
\usepackage{titletoc}
\usepackage{chngcntr} % Dùng để thiết lập lại cách đánh số caption,..
\usepackage{pdflscape} % Đưa các bảng có kích thước đặt theo chiều ngang giấy
\usepackage{afterpage}
\usepackage[ruled,vlined]{algorithm2e}  % Hỗ trợ viết các giải thuật
\usepackage{capt-of} % Cho phép sử dụng caption lớn đối với landscape page
\usepackage{multirow} % Merge cells
\usepackage{fancyhdr} % Cho phép tùy biến header và footer
% \usepackage[natbib,backend=biber,style=ieee]{biblatex} % Giúp chèn tài liệu tham khảo

\usepackage[font=small,labelfont=bf]{caption}

\usepackage{listings}
\usepackage{float}
\usepackage{subcaption}

\usepackage[nonumberlist, nopostdot, nogroupskip, acronym]{glossaries}
\usepackage{glossary-superragged}
\setglossarystyle{superraggedheaderborder}
\usepackage{setspace}
\usepackage{parskip}

% package content table
\usepackage{tocbasic}

\usepackage{blindtext}


% ===================================================

% \renewcommand{\bibname}{Danh_sach_tai_lieu_tham_khao} 
\usepackage[backend=bibtex,style=ieee]{biblatex}  %backend=biber is 'better'

\addbibresource{reference.bib} % chèn file chứa danh mục tài liệu tham khảo vào 

\include{lstlisting} % Phần này cho phép chèn code và formatting code như C, C++, Python

\makenoidxglossaries

\newglossaryentry{LLM}{
  type=\acronymtype,
  name={LLM},
  description={Large Language Model},
  first={Large Language Model (LLM)},
}
\newglossaryentry{RL}{
  type=\acronymtype,
  name={RL},
  description={Reinforcement Learning},
  first={Reinforcement Learning (RL)}
}
\newglossaryentry{BaH}{
  type=\acronymtype,
  name={BaH},
  description={Buy and Hold},
  first={Buy and Hold (BaH)}
}
\newglossaryentry{UP}{
  type=\acronymtype,
  name={UP},
  description={Universal Portfolios},
  first={Universal Portfolios (UP)}
}
\newglossaryentry{CORN}{
  type=\acronymtype,
  name={CORN},
  description={Correlation-driven Nonparametric},
  first={Correlation-driven Nonparametric (CORN)}
}
\newglossaryentry{ANTICOR}{
  type=\acronymtype,
  name={ANTICOR},
  description={Anti-correlation driven Nonparametric},
  first={Anti-correlation driven Nonparametric (ANTICOR)}
}
\newglossaryentry{MDP}{
  type=\acronymtype,
  name={MDP},
  description={Markov Decision Process},
  first={Markov Decision Process (MDP)}
}
\newglossaryentry{SAC}{
  type=\acronymtype,
  name={SAC},
  description={Soft Actor-Critic},
  first={Soft Actor-Critic (SAC)}
}
\newglossaryentry{PPO}{
  type=\acronymtype,
  name={PPO},
  description={Proximal Policy Optimization},
  first={Proximal Policy Optimization (PPO)}
}
\newglossaryentry{DDPG}{
  type=\acronymtype,
  name={DDPG},
  description={Deep Deterministic Policy Gradient},
  first={Deep Deterministic Policy Gradient (DDPG)}
}
\newglossaryentry{DQN}{
  type=\acronymtype,
  name={DQN},
  description={Deep Q-Network},
  first={Deep Q-Network (DQN)}
}
\newglossaryentry{LSTM}{
  type=\acronymtype,
  name={LSTM},
  description={Long Short-Term Memory},
  first={Long Short-Term Memory (LSTM)}
}
\newglossaryentry{SR}{
  type=\acronymtype,
  name={SR},
  description={Sharpe Ratio},
  first={Sharpe Ratio (SR)}
}
\newglossaryentry{BERT}{
  type=\acronymtype,
  name={BERT},
  description={Bidirectional Encoder Representations from Transformers},
  first={Bidirectional Encoder Representations from Transformers (BERT)}
}
\newglossaryentry{POMDP}{
  type=\acronymtype,
  name={POMDP},
  description={Partially Observable Markov Decision Process},
  first={Partially Observable Markov Decision Proces (POMDP)}
}
\newglossaryentry{OHLCV}{
  type=\acronymtype,
  name={OHLCV},
  description={Open, High, Low, Close, Volume},
  first={Open, High, Low, Close, Volume (OHLCV)}
}
\newglossaryentry{LSA}{
  type=\acronymtype,
  name={LSA},
  description={Latent Semantic Analysis},
  first={Latent Semantic Analysis (LSA)}
}
\newglossaryentry{MACD}{
  type=\acronymtype,
  name={MACD},
  description={Moving Average Convergence Divergence},
  first={Moving Average Convergence Divergence (MACD)}
}
\newglossaryentry{BB}{
  type=\acronymtype,
  name={BB},
  description={Bollinger Bands},
  first={Bollinger Bands (BB)}
}
\newglossaryentry{RSI}{
  type=\acronymtype,
  name={RSI},
  description={Relative Strength Index},
  first={Relative Strength Index (RSI)}
}
\newglossaryentry{CCI}{
  type=\acronymtype,
  name={CCI},
  description={Commodity Channel Index},
  first={Commodity Channel Index (CCI)}
}
\newglossaryentry{EMA}{
  type=\acronymtype,
  name={EMA},
  description={Exponential Moving Average},
  first={Exponential Moving Average (EMA)}
}
\newglossaryentry{DMI}{
  type=\acronymtype,
  name={DX},
  description={Directional Movement Index},
  first={Directional Movement Index (DMI)}
}
\newglossaryentry{ADX}{
  type=\acronymtype,
  name={ADX},
  description={Average Directional Movement Index},
  first={Average Directional Movement Index (ADX)}
}
\newglossaryentry{CPV}{
  type=\acronymtype,
  name={CPV},
  description={Cumulative Portfolio Value},
  first={Cumulative Portfolio Value (CPV)}
}
\newglossaryentry{MDD}{
  type=\acronymtype,
  name={MDD},
  description={Maximum Drawdown},
  first={Maximum Drawdown (MDD)}
}
\newglossaryentry{CAGR}{
  type=\acronymtype,
  name={CAGR},
  description={Compound Annual Growth Rate},
  first={Compound Annual Growth Rate (CAGR)}
}


% ===================================================


\fancypagestyle{plain}{%
\fancyhf{} % clear all header and footer fields
\fancyfoot[RO,RE]{\thepage} %RO=right odd, RE=right even
\renewcommand{\headrulewidth}{0pt}
\renewcommand{\footrulewidth}{0pt}}

\setlength{\headheight}{10pt}

\def \TITLE{GRADUATION THESIS}
\def \AUTHOR{Trần Văn A}

% ===================================================
\titleformat{\chapter}[hang]{\centering\bfseries}{CHAPTER \thechapter.\ }{0pt}{}[]

\titleformat 
    {\chapter} % command
    [hang] % shape
    {\centering\bfseries} % format
    {CHAPTER \thechapter.\ } % label
    {0pt} %sep
    {} % before
    [] % after
\titlespacing*{\chapter}{0pt}{-20pt}{20pt}

\titleformat
    {\section} % command
    [hang] % shape
    {\bfseries} % format
    {\thechapter.\arabic{section}\ \ \ \ } % label
    {0pt} %sep
    {} % before
    [] % after
\titlespacing{\section}{0pt}{\parskip}{0.5\parskip}

\titleformat
    {\subsection} % command
    [hang] % shape
    {\bfseries} % format
    {\thechapter.\arabic{section}.\arabic{subsection}\ \ \ \ } % label
    {0pt} %sep
    {} % before
    [] % after
\titlespacing{\subsection}{30pt}{\parskip}{0.5\parskip}

\renewcommand\thesubsubsection{\alph{subsubsection}}
\titleformat
    {\subsubsection} % command
    [hang] % shape
    {\bfseries} % format
    {\alph{subsubsection}, \ } % label
    {0pt} %sep
    {} % before
    [] % after
\titlespacing{\subsubsection}{50pt}{\parskip}{0.5\parskip}

% \newcommand{\titlesize}{\fontsize{18pt}{23pt}\selectfont}
% \newcommand{\subtitlesize}{\fontsize{16pt}{21pt}\selectfont}
% \titleclass{\part}{top}
% \titleformat{\part}[display]
%   {\normalfont\huge\bfseries}{\centering}{20pt}{\Huge\centering}
% \titlespacing{\part}{0pt}{em}{1em}
% \titlespacing{\section}{0pt}{\parskip}{0.5\parskip}
% \titlespacing{\subsection}{0pt}{\parskip}{0.5\parskip}
% \titlespacing{\subsubsection}{0pt}{\parskip}{0.5\parskip}



% ===================================================
\usepackage{hyperref}
\hypersetup{pdfborder = {0 0 0}}
\hypersetup{pdftitle={\TITLE},
	pdfauthor={\AUTHOR}}
	
\usepackage[all]{hypcap} % Cho phép tham chiếu chính xác đến hình ảnh và bảng biểu

\graphicspath{{figures/}{../figures/}} % Thư mục chứa các hình ảnh

\counterwithin{figure}{chapter} % Đánh số hình ảnh kèm theo chapter. Ví dụ: Hình 1.1, 1.2,..

\title{\bf \TITLE}
\author{\AUTHOR}

\setcounter{secnumdepth}{3} % Cho phép subsubsection trong report
% \setcounter{tocdepth}{3} % Chèn subsubsection vào bảng mục lục

\theoremstyle{definition}
\newtheorem{example}{Example}[chapter] % Định nghĩa môi trường ví dụ

\onehalfspacing
\setlength{\parskip}{6pt}
\setlength{\parindent}{15pt}



% =========================== BODY ===============
\begin{document}
% \newgeometry{top=2cm, bottom=2cm, left=2cm, right=2cm}
\subfile{Cover} % Phần bìa
% \restoregeometry

% ===================================================
\pagestyle{empty} % Header và footer rỗng
%\newpage
%\pagenumbering{gobble} % Xóa page numbering ở cuối trang
%\subfile{chapters/0_1_subject.tex}

% \pagestyle{empty} % Header và footer rỗng
\newpage
\pagenumbering{gobble} % Xóa page numbering ở cuối trang
\subfile{content/0_2_acknowledgment.tex}

% \pagestyle{empty} % Header và footer rỗng
\newpage
\pagenumbering{gobble} % Xóa page numbering ở cuối trang
\subfile{content/0_3_Abstract.tex}


% ===================================================
% \pagestyle{empty} % Header và footer rỗng
\newpage
\pagenumbering{gobble} % Xóa page numbering ở cuối trang
\renewcommand*\contentsname{TABLE OF CONTENTS}
\titlecontents{chapter}
    [0.0cm]             % left margin
    {\bfseries\vspace{0.3cm}}                  % above code
    {{\bfseries{\scshape} CHAPTER \thecontentslabel.\ }} % numbered format
    {}         % unnumbered format
    {\titlerule*[0.3pc]{.}\contentspage}         % filler-page-format, e.g dots
    
\titlecontents{section}
    [0.0cm]             % left margin
    {\vspace{0.3cm}}                  % above code
    {\thecontentslabel \ } % numbered format
    {}         % unnumbered format
    {\titlerule*[0.3pc]{.}\contentspage}         % filler-page-format, e.g dots
    
\titlecontents{subsection}
    [1.0cm]             % left margin
    {\vspace{0.3cm}}                  % above code
    {\thecontentslabel \ } % numbered format
    {}         % unnumbered format
    {\titlerule*[0.3pc]{.}\contentspage}         % filler-page-format, e.g dots

 % Tạo mục lục tự động
\addtocontents{toc}{\protect\thispagestyle{empty}}
\tableofcontents 
\thispagestyle{empty}
\cleardoublepage

\pagenumbering{roman}
%Tạo danh mục hình vẽ.
\renewcommand{\listfigurename}{LIST OF FIGURES}
{\let\oldnumberline\numberline
\renewcommand{\numberline}{Figure~\oldnumberline}
\listoffigures} 
% \phantomsection\addcontentsline{toc}{section}{\numberline {} DANH MỤC HÌNH VẼ}
\newpage


 %Tạo danh mục bảng biểu.
\renewcommand{\listtablename}{LIST OF TABLES}
{\let\oldnumberline\numberline
\renewcommand{\numberline}{Table~\oldnumberline}
\listoftables}
% \phantomsection\addcontentsline{toc}{section}{\numberline {} DANH MỤC BẢNG BIỂU}

\glsaddall 
% \renewcommand*{\glossaryname}{Danh sách thuật ngữ}
\renewcommand*{\acronymname}{LIST OF ABBREVIATIONS}
\renewcommand*{\entryname}{Abbreviation}
\renewcommand*{\descriptionname}{Definition}
\printnoidxglossaries
% \phantomsection\addcontentsline{toc}{section}{\numberline {} DANH MỤC THUẬT NGỮ VÀ TỪ VIẾT TẮT}

\renewcommand\appendixname{APPENDIX}
\renewcommand\appendixpagename{APPENDIX}
\renewcommand\appendixtocname{APPENDIX}

\renewcommand{\figurename}{Figure}
\renewcommand{\tablename}{Table}
\renewcommand{\chaptername}{CHAPTER}

% ===================================================


\newpage
\pagenumbering{arabic}

\pagestyle{fancy}
\fancyhf{}
\fancyhead[RE, LO]{\leftmark}
%\fancyhead[LE]{\rightmark}
\fancyfoot[RE, LO]{\thepage}

\chapter{INTRODUCTION}
\subfile{content/1_introduction} % Phần mở đầu

\newpage
%\pagestyle{fancy} % Áp dụng header và footer
\chapter{LITERATURE REVIEW}
\subfile{content/2_literature_review}


\newpage
%\pagestyle{fancy} % Áp dụng header và footer
\chapter{METHODOLOGY}
\subfile{content/3_methodology}

\newpage
%\pagestyle{fancy} % Áp dụng header và footer
\chapter{THEORETICAL ANALYSIS}
\subfile{content/4_theoretical_analysis}

\newpage
%\pagestyle{fancy} % Áp dụng header và footer
\chapter{NUMERICAL RESULTS}
\subfile{content/5_numerical_results}

\newpage
\chapter{CONCLUSIONS}
\subfile{content/6_conclusions}

\newpage
%\pagestyle{fancy} % Áp dụng header và footer
\chapter*{SHORT NOTICES ON REFERENCE} %Kết luận và hướng phát triển}
\label{chapter:reference}
\subfile{content/7_reference}

% ===================================================
\newpage
\renewcommand\bibname{REFERENCE}
\printbibliography
\phantomsection\addcontentsline{toc}{chapter}{REFERENCE}

\appendixpage
\appendix
\addappheadtotoc

\titleformat{\chapter}[hang]{\centering\bfseries}{ \thechapter.\ }{0pt}{}[]
\titlespacing*{\chapter}{0pt}{-20pt}{20pt}

\titlecontents{chapter}
    [0.0cm]             % left margin
    {\bfseries\vspace{0.3cm}}                  % above code
    {{\bfseries{\scshape} \thecontentslabel.\ }} % numbered format
    {}         % unnumbered format
    {\titlerule*[0.3pc]{.}\contentspage} 
    
\chapter{GRADUATION THESIS GUIDANCE}
\subfile{content/appendix_a}

\newpage
\chapter{USE CASE DESCRIPTIONS}
\subfile{content/appendix_b}

\end{document}
