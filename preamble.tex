\usepackage{parskip}
\usepackage{fontspec}
\usepackage[main=english,vietnamese]{babel}
\setmainfont{Times New Roman}
\usepackage{scrextend}
\usepackage[top=2cm, bottom=2cm, left=3.5cm, right=2.5cm]{geometry}
\usepackage{xurl}
\usepackage{appendix}
\usepackage{xcolor}
\usepackage{outlines}
\usepackage{graphicx} % Cho phép chèn hỉnh ảnh
\usepackage{fancybox} % Tạo khung box
\usepackage{indentfirst} % Thụt đầu dòng ở dòng đầu tiên trong đoạn
\usepackage{amsthm} % Cho phép thêm các môi trường định nghĩa
\usepackage{latexsym} % Các kí hiệu toán học
\usepackage{amsmath} % Hỗ trợ một số biểu thức toán học
\usepackage{amssymb} % Bổ sung thêm kí hiệu về toán học
\usepackage{amsbsy} % Hỗ trợ các kí hiệu in đậm
\usepackage{array} % Tạo bảng array
\usepackage{enumitem} % Cho phép thay đổi kí hiệu của list
\usepackage{subfiles} % Chèn các file nhỏ, giúp chia các chapter ra nhiều file hơn
\usepackage{titlesec} % Giúp chỉnh sửa các tiêu đề, đề mục như chương, phần,..
\usepackage{titletoc}
\usepackage{chngcntr} % Dùng để thiết lập lại cách đánh số caption,..
\usepackage{pdflscape} % Đưa các bảng có kích thước đặt theo chiều ngang giấy
\usepackage{afterpage}
\usepackage[ruled,vlined]{algorithm2e}  % Hỗ trợ viết các giải thuật
\usepackage{capt-of} % Cho phép sử dụng caption lớn đối với landscape page
\usepackage{multirow} % Merge cells
\usepackage{fancyhdr} % Cho phép tùy biến header và footer
% \usepackage[natbib,backend=biber,style=ieee]{biblatex} % Giúp chèn tài liệu tham khảo

\usepackage[font=small,labelfont=bf]{caption}

\usepackage{listings}
\usepackage{float}
\usepackage{subcaption}

\usepackage[nonumberlist, nopostdot, nogroupskip, acronym]{glossaries}
\usepackage{glossary-superragged}
\setglossarystyle{superraggedheaderborder}
\usepackage{setspace}

% package content table
\usepackage{tocbasic}

\usepackage{blindtext}
% ===================================================

% \renewcommand{\bibname}{Danh_sach_tai_lieu_tham_khao} 
\usepackage[backend=bibtex,style=ieee]{biblatex}  %backend=biber is 'better'

\addbibresource{reference.bib} % chèn file chứa danh mục tài liệu tham khảo vào 

\include{lstlisting} % Phần này cho phép chèn code và formatting code như C, C++, Python

\makenoidxglossaries

\newglossaryentry{LLM}{
  type=\acronymtype,
  name={LLM},
  description={Large Language Model},
  first={Large Language Model (LLM)},
}
\newglossaryentry{RL}{
  type=\acronymtype,
  name={RL},
  description={Reinforcement Learning},
  first={Reinforcement Learning (RL)}
}
\newglossaryentry{BaH}{
  type=\acronymtype,
  name={BaH},
  description={Buy and Hold},
  first={Buy and Hold (BaH)}
}
\newglossaryentry{UP}{
  type=\acronymtype,
  name={UP},
  description={Universal Portfolios},
  first={Universal Portfolios (UP)}
}
\newglossaryentry{CORN}{
  type=\acronymtype,
  name={CORN},
  description={Correlation-driven Nonparametric},
  first={Correlation-driven Nonparametric (CORN)}
}
\newglossaryentry{ANTICOR}{
  type=\acronymtype,
  name={ANTICOR},
  description={Anti-correlation driven Nonparametric},
  first={Anti-correlation driven Nonparametric (ANTICOR)}
}


% ===================================================

\def \TITLE{GRADUATION THESIS}
\def \AUTHOR{Trần Văn A}

\fancypagestyle{plain}{%
\fancyhf{} % clear all header and footer fields
\fancyfoot[RO,RE]{\thepage} %RO=right odd, RE=right even
\renewcommand{\headrulewidth}{0pt}
\renewcommand{\footrulewidth}{0pt}}

\setlength{\headheight}{10pt}
% ===================================================
\titleformat{\chapter}[hang]{\centering\bfseries}{CHAPTER \thechapter.\ }{0pt}{}[]

\titleformat 
    {\chapter} % command
    [hang] % shape
    {\centering\bfseries} % format
    {CHAPTER \thechapter.\ } % label
    {0pt} %sep
    {} % before
    [] % after
\titlespacing*{\chapter}{0pt}{-20pt}{20pt}

\titleformat
    {\section} % command
    [hang] % shape
    {\bfseries} % format
    {\thechapter.\arabic{section}\ \ \ \ } % label
    {0pt} %sep
    {} % before
    [] % after
\titlespacing{\section}{0pt}{\parskip}{0.5\parskip}

\titleformat
    {\subsection} % command
    [hang] % shape
    {\bfseries} % format
    {\thechapter.\arabic{section}.\arabic{subsection}\ \ \ \ } % label
    {0pt} %sep
    {} % before
    [] % after
\titlespacing{\subsection}{30pt}{\parskip}{0.5\parskip}

\renewcommand\thesubsubsection{\alph{subsubsection}}
\titleformat
    {\subsubsection} % command
    [hang] % shape
    {\bfseries} % format
    {\alph{subsubsection}, \ } % label
    {0pt} %sep
    {} % before
    [] % after
\titlespacing{\subsubsection}{50pt}{\parskip}{0.5\parskip}

% \newcommand{\titlesize}{\fontsize{18pt}{23pt}\selectfont}
% \newcommand{\subtitlesize}{\fontsize{16pt}{21pt}\selectfont}
% \titleclass{\part}{top}
% \titleformat{\part}[display]
%   {\normalfont\huge\bfseries}{\centering}{20pt}{\Huge\centering}
% \titlespacing{\part}{0pt}{em}{1em}
% \titlespacing{\section}{0pt}{\parskip}{0.5\parskip}
% \titlespacing{\subsection}{0pt}{\parskip}{0.5\parskip}
% \titlespacing{\subsubsection}{0pt}{\parskip}{0.5\parskip}



% ===================================================
\usepackage{hyperref}
\hypersetup{pdfborder = {0 0 0}}
\hypersetup{pdftitle={\TITLE},
	pdfauthor={\AUTHOR}}
	
\usepackage[all]{hypcap} % Cho phép tham chiếu chính xác đến hình ảnh và bảng biểu

\graphicspath{{figures/}{../figures/}} % Thư mục chứa các hình ảnh

\counterwithin{figure}{chapter} % Đánh số hình ảnh kèm theo chapter. Ví dụ: Hình 1.1, 1.2,..

\title{\bf \TITLE}
\author{\AUTHOR}

\setcounter{secnumdepth}{3} % Cho phép subsubsection trong report
% \setcounter{tocdepth}{3} % Chèn subsubsection vào bảng mục lục

\theoremstyle{definition}
\newtheorem{example}{Example}[chapter]

\onehalfspacing
\setlength{\parskip}{6pt}
\setlength{\parindent}{15pt}
